% oriented to output, that is, what fonts to use for printing characters
\usepackage[T1]{fontenc}

% allows the user to input accented characters directly from the keyboard
\usepackage[utf8]{inputenc}

% for making database like glossary, acronyms and notation
%\usepackage{datagidx} 

% for mathematical things ... no need for amsmath
\usepackage{mathtools}
\usepackage{amssymb}
%\usepackage{amsthm}
\DeclareMathOperator\arctanh{arctanh}

% for proof environment
%\usepackage{amsthm}

% package for bold math
\usepackage{bm}

% caption center
\usepackage[center]{caption}
% set spaces between floats and text below it
%\setlength{\belowcaptionskip}{0mm}
%\setlength{\intextsep}{0mm}
%\setlength{\textfloatsep}{0mm}

% not numbering equations that are not mentioned
%\mathtoolsset{showonlyrefs,showmanualtags}

% for subfigures
\usepackage{subfigure}

% subfigure numbering style
%\renewcommand{\thesubfigure}{\thefigure(\alph{subfigure})}
%\makeatletter
%	\renewcommand{\p@subfigure}{}
%	\renewcommand{\@thesubfigure}{\thesubfigure:\hskip\subfiglabelskip} 
%\makeatother

% change line spacing
\usepackage{setspace}

% used for symbol newmoon
\usepackage{wasysym}

% for hyperlinks
\usepackage{hyperref}

% these packages must come after color
% for technical drawings
\usepackage{tikz}
\usepackage{tikz-3dplot}
\usepackage{tkz-euclide}
\usetkzobj{all}
\usetikzlibrary{calc,decorations.markings,,intersections,shapes.misc}

% Allows including images
\usepackage{graphicx}

% multiple rows/columns in table
\usepackage{multirow}

% so figures don't get out of place: used to recognize FloatBarrier
\usepackage{placeins}

\definecolor{dkgreen}{rgb}{0,0.6,0}

% for hyperreferences: must come before glossary
\usepackage{hyperref}
% hyperref options
\hypersetup{
	colorlinks=true,
	linkcolor=red,
}

%% helper macros
%% THIS IS FOR FIGURE OF PROJECTION OF SPHERE IN PLANE %%
\newcommand\pgfmathsinandcos[3]{%
  \pgfmathsetmacro#1{sin(#3)}%
  \pgfmathsetmacro#2{cos(#3)}%
}
\newcommand\LongitudePlane[3][current plane]{%
  \pgfmathsinandcos\sinEl\cosEl{#2} % elevation
  \pgfmathsinandcos\sint\cost{#3} % azimuth
  \tikzset{#1/.estyle={cm={\cost,\sint*\sinEl,0,\cosEl,(0,0)}}}
}
\newcommand\LatitudePlane[3][current plane]{%
  \pgfmathsinandcos\sinEl\cosEl{#2} % elevation
  \pgfmathsinandcos\sint\cost{#3} % latitude
  \pgfmathsetmacro\yshift{\cosEl*\sint}
  \tikzset{#1/.estyle={cm={\cost,0,0,\cost*\sinEl,(0,\yshift)}}} %
}
\newcommand\DrawLongitudeCircle[2][1]{
  \LongitudePlane{\angEl}{#2}
  \tikzset{current plane/.prefix style={scale=#1}}
   % angle of "visibility"
  \pgfmathsetmacro\angVis{atan(sin(#2)*cos(\angEl)/sin(\angEl))} %
  \draw[current plane] (\angVis:1) arc (\angVis:\angVis+180:1);
  \draw[current plane,dashed] (\angVis-180:1) arc (\angVis-180:\angVis:1);
}
\newcommand\DrawLatitudeCircle[2][1]{
  \LatitudePlane{\angEl}{#2}
  \tikzset{current plane/.prefix style={scale=#1}}
  \pgfmathsetmacro\sinVis{sin(#2)/cos(#2)*sin(\angEl)/cos(\angEl)}
  % angle of "visibility"
  \pgfmathsetmacro\angVis{asin(min(1,max(\sinVis,-1)))}
  \draw[current plane] (\angVis:1) arc (\angVis:-\angVis-180:1);
  \draw[current plane,dashed] (180-\angVis:1) arc (180-\angVis:\angVis:1);
}
%% THIS IS FOR FIGURE OF PROJECTION OF SPHERE IN PLANE %%

%%% document-wide tikz options and styles
%\tikzset{%
%  >=latex, % option for nice arrows
%  inner sep=0pt,%
%  outer sep=2pt,%
%  mark coordinate/.style={inner sep=0pt,outer sep=0pt,minimum size=3pt,
%    fill=black,circle}%
%}